\documentclass{article}
\usepackage[utf8]{inputenc}
\usepackage[magyar]{babel}
\usepackage[T1]{fontenc}
\usepackage[nomath]{lmodern}
\usepackage[hmargin=2cm]{geometry}
\usepackage{caption}
\usepackage{pdflscape}
\usepackage{tikz}
\tikzset{every picture/.style={/utils/exec={\sffamily}}}
%\usetikzlibrary{calc, shadings, shadows, shapes.arrows, shapes.symbols}
\usetikzlibrary{backgrounds,calc,shadings,shapes.arrows,shapes.symbols,shadows}

% =============================================================
% Csomo definicio van itt, ez ne riasszon el, ez fix definicio.
% A lenyeg a legaljan van, ahol felhasznaljuk
% =============================================================

\definecolor{myblue}{rgb}{0.5, 0.5, 1}
\definecolor{mymagenta}{rgb}{1, 0, 1}

\makeatletter
\pgfkeys{/pgf/.cd,
  parallelepiped offset x/.initial=2mm,
  parallelepiped offset y/.initial=2mm
}
\pgfdeclareshape{parallelepiped}
{
  \inheritsavedanchors[from=rectangle] % this is nearly a rectangle
  \inheritanchorborder[from=rectangle]
  \inheritanchor[from=rectangle]{north}
  \inheritanchor[from=rectangle]{north west}
  \inheritanchor[from=rectangle]{north east}
  \inheritanchor[from=rectangle]{center}
  \inheritanchor[from=rectangle]{west}
  \inheritanchor[from=rectangle]{east}
  \inheritanchor[from=rectangle]{mid}
  \inheritanchor[from=rectangle]{mid west}
  \inheritanchor[from=rectangle]{mid east}
  \inheritanchor[from=rectangle]{base}
  \inheritanchor[from=rectangle]{base west}
  \inheritanchor[from=rectangle]{base east}
  \inheritanchor[from=rectangle]{south}
  \inheritanchor[from=rectangle]{south west}
  \inheritanchor[from=rectangle]{south east}
  \backgroundpath{
    % store lower right in xa/ya and upper right in xb/yb
    \southwest \pgf@xa=\pgf@x \pgf@ya=\pgf@y
    \northeast \pgf@xb=\pgf@x \pgf@yb=\pgf@y
    \pgfmathsetlength\pgfutil@tempdima{\pgfkeysvalueof{/pgf/parallelepiped
      offset x}}
    \pgfmathsetlength\pgfutil@tempdimb{\pgfkeysvalueof{/pgf/parallelepiped
      offset y}}
    \def\ppd@offset{\pgfpoint{\pgfutil@tempdima}{\pgfutil@tempdimb}}
    \pgfpathmoveto{\pgfqpoint{\pgf@xa}{\pgf@ya}}
    \pgfpathlineto{\pgfqpoint{\pgf@xb}{\pgf@ya}}
    \pgfpathlineto{\pgfqpoint{\pgf@xb}{\pgf@yb}}
    \pgfpathlineto{\pgfqpoint{\pgf@xa}{\pgf@yb}}
    \pgfpathclose
    \pgfpathmoveto{\pgfqpoint{\pgf@xb}{\pgf@ya}}
    \pgfpathlineto{\pgfpointadd{\pgfpoint{\pgf@xb}{\pgf@ya}}{\ppd@offset}}
    \pgfpathlineto{\pgfpointadd{\pgfpoint{\pgf@xb}{\pgf@yb}}{\ppd@offset}}
    \pgfpathlineto{\pgfpointadd{\pgfpoint{\pgf@xa}{\pgf@yb}}{\ppd@offset}}
    \pgfpathlineto{\pgfqpoint{\pgf@xa}{\pgf@yb}}
    \pgfpathmoveto{\pgfqpoint{\pgf@xb}{\pgf@yb}}
    \pgfpathlineto{\pgfpointadd{\pgfpoint{\pgf@xb}{\pgf@yb}}{\ppd@offset}}
  }
}
\makeatother

\tikzset{l3 switch/.style={
    parallelepiped,fill=switch, draw=white,
    minimum width=0.75cm,
    minimum height=0.75cm,
    parallelepiped offset x=1.75mm,
    parallelepiped offset y=1.25mm,
    path picture={
      \node[fill=white,
        circle,
        minimum size=6pt,
        inner sep=0pt,
        append after command={
          \pgfextra{
            \foreach \angle in {0,45,...,360}
            \draw[-latex,fill=white] (\tikzlastnode.\angle)--++(\angle:2.25mm);
          }
        }
      ] 
       at ([xshift=-0.75mm,yshift=-0.5mm]path picture bounding box.center){};
    }
  },
  ports/.style={
    line width=0.3pt,
    top color=gray!20,
    bottom color=gray!80
  },
  rack switch/.style={
    parallelepiped,fill=white, draw,
    minimum width=1.25cm,
    minimum height=0.25cm,
    parallelepiped offset x=2mm,
    parallelepiped offset y=1.25mm,
    xscale=-1,
    path picture={
      \draw[top color=gray!5,bottom color=gray!40] (path picture bounding box.south west) rectangle (path picture bounding box.north east);
      \coordinate (A-west) at ([xshift=-0.2cm]path picture bounding box.west);
      \coordinate (A-center) at ($(path picture bounding box.center)!0!(path picture bounding box.south)$);
      \foreach \x in {0.275,0.525,0.775}{
        \draw[ports]([yshift=-0.05cm]$(A-west)!\x!(A-center)$) rectangle +(0.1,0.05);
        \draw[ports]([yshift=-0.125cm]$(A-west)!\x!(A-center)$) rectangle +(0.1,0.05);
      }
      \coordinate (A-east) at (path picture bounding box.east);
      \foreach \x in {0.085,0.21,0.335,0.455,0.635,0.755,0.875,1}{
        \draw[ports]([yshift=-0.1125cm]$(A-east)!\x!(A-center)$) rectangle +(0.05,0.1);
      }
    }
  },
  server/.style={
    parallelepiped,
    fill=white, draw,
    minimum width=0.35cm,
    minimum height=0.75cm,
    parallelepiped offset x=3mm,
    parallelepiped offset y=2mm,
    xscale=-1,
    path picture={
      \draw[top color=myblue!80,bottom color=myblue!40] (path picture bounding box.south west) rectangle (path picture bounding box.north east);
      \coordinate (A-center) at ($(path picture bounding box.center)!0!(path picture bounding box.south)$);
      \coordinate (A-west) at ([xshift=-0.575cm]path picture bounding box.west);
      \draw[ports]([yshift=0.1cm]$(A-west)!0!(A-center)$) rectangle +(0.2,0.065);
      \draw[ports]([yshift=0.01cm]$(A-west)!0.085!(A-center)$) rectangle +(0.15,0.05);
      \fill[black]([yshift=-0.35cm]$(A-west)!-0.1!(A-center)$) rectangle +(0.235,0.0175);
      \fill[black]([yshift=-0.385cm]$(A-west)!-0.1!(A-center)$) rectangle +(0.235,0.0175);
      \fill[black]([yshift=-0.42cm]$(A-west)!-0.1!(A-center)$) rectangle +(0.235,0.0175);
    }  
  },
  serverhidden/.style={
    parallelepiped,
    fill=white, draw,
    minimum width=0.35cm,
    minimum height=0.75cm,
    parallelepiped offset x=3mm,
    parallelepiped offset y=2mm,
    xscale=-1,
    path picture={
      \draw[top color=lightgray!5,bottom color=lightgray!20] (path picture bounding box.south west) rectangle (path picture bounding box.north east);
      \coordinate (A-center) at ($(path picture bounding box.center)!0!(path picture bounding box.south)$);
      \coordinate (A-west) at ([xshift=-0.575cm]path picture bounding box.west);
      \draw[ports]([yshift=0.1cm]$(A-west)!0!(A-center)$) rectangle +(0.2,0.065);
      \draw[ports]([yshift=0.01cm]$(A-west)!0.085!(A-center)$) rectangle +(0.15,0.05);
      \fill[lightgray]([yshift=-0.35cm]$(A-west)!-0.1!(A-center)$) rectangle +(0.235,0.0175);
      \fill[lightgray]([yshift=-0.385cm]$(A-west)!-0.1!(A-center)$) rectangle +(0.235,0.0175);
      \fill[lightgray]([yshift=-0.42cm]$(A-west)!-0.1!(A-center)$) rectangle +(0.235,0.0175);
    }  
  },
}


% Styles for interfaces and edge labels
\tikzset{%
  interface/.style={draw, rectangle, rounded corners, font=\LARGE\sffamily},
  ethernet/.style={interface, fill=yellow!50},% ethernet interface
  mgmtether/.style={interface, fill=black!50},% mgmtether interface
  vlan/.style={interface, fill=red!50},% vlan interface
  vpn/.style={interface, fill=red!50},% vpn interface
  serial/.style={interface, fill=green!70},% serial interface
  speed/.style={sloped, anchor=south, font=\large\sffamily},% line speed at edge
  internal_arrow/.style={draw, shape=single arrow, single arrow head extend=4mm,
    minimum height=1.7cm, minimum width=3mm, white, fill=myblue!20,
    drop shadow={opacity=.8, fill=myblue!50!black}, font=\tiny}% inroute / outroute arrows
}

% The router icon
\newcommand*{\router}[2]{
 \begin{tikzpicture}
  \coordinate (ll) at (-3,0);
  \coordinate (lr) at (3,0);
  \coordinate (ul) at (-3,2);
  \coordinate (ur) at (3,2);
  \shade [shading angle=90, left color=black!40!myblue, right color=white] (ll) arc (-180:-60:3cm and .75cm) -- +(0,2) arc (-60:-180:3cm and .75cm) -- cycle;
  \shade [shading angle=270, right color=black!40!myblue, left color=white!50] (lr) arc (0:-60:3cm and .75cm) -- +(0,2) arc (-60:0:3cm and .75cm) -- cycle;
  \draw [thick] (ll) arc (-180:0:3cm and .75cm) -- (ur) arc (0:-180:3cm and .75cm) -- cycle;
  \draw [thick, shade, upper left=myblue!30!black, lower left=myblue!80!white, upper right=myblue!80!white, lower right=white] (ul) arc (-180:180:3cm and .75cm);
  \node at (0,0.7){\color{blue!70!black}\huge #1};% The name of the router
  \node at (0,-0.1){\color{blue!70!black}\huge #2};% The IP address of the router
  % The four arrows, symbols for incoming and outgoing routes:
  \newcommand*{\shift}{1.3cm}% For placing the arrows later
  \begin{scope}[yshift=2cm, yscale=0.28, transform shape]
    \node[internal_arrow, rotate=45, xshift=\shift] {\strut};
    \node[internal_arrow, rotate=-45, xshift=-\shift] {\strut};
    \node[internal_arrow, rotate=-135, xshift=\shift] {\strut};
    \node[internal_arrow, rotate=135, xshift=-\shift] {\strut};
  \end{scope}
 \end{tikzpicture}
}

\newcommand*{\routerhidden}[2]{
 \begin{tikzpicture}
  \coordinate (ll) at (-3,0);
  \coordinate (lr) at (3,0);
  \coordinate (ul) at (-3,2);
  \coordinate (ur) at (3,2);
  \shade [shading angle=90, left color=gray!40!lightgray, right color=white] (ll) arc (-180:-60:3cm and .75cm) -- +(0,2) arc (-60:-180:3cm and .75cm) -- cycle;
  \shade [shading angle=270, right color=gray!40!lightgray, left color=white!50] (lr) arc (0:-60:3cm and .75cm) -- +(0,2) arc (-60:0:3cm and .75cm) -- cycle;
  \draw [thick] (ll) arc (-180:0:3cm and .75cm) -- (ur) arc (0:-180:3cm and .75cm) -- cycle;
  \draw [thick, shade, upper left=lightgray!30!gray, lower left=lightgray!80!white, upper right=lightgray!80!white, lower right=white] (ul) arc (-180:180:3cm and .75cm);
  \node at (0,0.7){\color{lightgray!70!gray}\huge #1};% The name of the router
  \node at (0,-0.1){\color{lightgray!70!gray}\huge #2};% The IP address of the router
  % The four arrows, symbols for incoming and outgoing routes:
  \newcommand*{\shift}{1.3cm}% For placing the arrows later
  \begin{scope}[yshift=2cm, yscale=0.28, transform shape]
    \node[internal_arrow, rotate=45, xshift=\shift] {\strut};
    \node[internal_arrow, rotate=-45, xshift=-\shift] {\strut};
    \node[internal_arrow, rotate=-135, xshift=\shift] {\strut};
    \node[internal_arrow, rotate=135, xshift=-\shift] {\strut};
  \end{scope}
 \end{tikzpicture}
}


% The switch icon
\newcommand*{\switch}[2]{
 \begin{tikzpicture}
  \coordinate (ll) at (-3, 0.5);
  \draw [thick, shade, upper left=myblue!30!black, lower left=myblue!80!white, upper right=myblue!80!white, lower right=white] (ll) rectangle (5, 3.4);
  \node at (0.5, 1.8){\color{blue!70!black}\huge #1};% The name of the switch
  \node at (0.5, 1.0){\color{blue!70!black}\huge #2};% The IP address of the switch
  \begin{scope}[yshift=2cm, yscale=0.28, transform shape]
    \node[internal_arrow, rotate=0, xshift=2cm, yshift=4.2cm] {\strut};
    \node[internal_arrow, rotate=0, xshift=2cm, yshift=2.2cm] {\strut};
    \node[internal_arrow, rotate=180, xshift=0cm, yshift=-3.2cm] {\strut};
    \node[internal_arrow, rotate=180, xshift=0cm, yshift=-1.2cm] {\strut};
  \end{scope}
 \end{tikzpicture}
}

\newcommand*{\switchhidden}[2]{
 \begin{tikzpicture}
  \coordinate (ll) at (-3, 0.5);
  \draw [thick, shade, upper left=gray!30!lightgray, lower left=lightgray!80!white, upper right=lightgray!80!white, lower right=white] (ll) rectangle (5, 3.4);
  \node at (0.5, 1.8){\color{gray!70!lightgray}\huge #1};% The name of the switch
  \node at (0.5, 1.0){\color{gray!70!lightgray}\huge #2};% The IP address of the switch
  \begin{scope}[yshift=2cm, yscale=0.28, transform shape]
    \node[internal_arrow, rotate=0, xshift=2cm, yshift=4.2cm] {\strut};
    \node[internal_arrow, rotate=0, xshift=2cm, yshift=2.2cm] {\strut};
    \node[internal_arrow, rotate=180, xshift=0cm, yshift=-3.2cm] {\strut};
    \node[internal_arrow, rotate=180, xshift=0cm, yshift=-1.2cm] {\strut};
  \end{scope}
 \end{tikzpicture}
}

% The server icon
\newcommand*{\server}[2] {
  \begin{tikzpicture}
    \node[server, scale=10]{};
    \node at (-2.5, 0.2){\color{blue!70!myblue}\huge #1};
    \node at (-2.5, -1.2){\color{blue!70!myblue}\huge #2};
  \end{tikzpicture}
}

\newcommand*{\serverhidden}[2] {
  \begin{tikzpicture}
    \node[serverhidden, scale=10]{};
    \node at (-2.5, 0.2){\color{lightgray!10!gray}\huge #1};
    \node at (-2.5, -1.2){\color{lightgray!10!gray}\huge #2};
  \end{tikzpicture}
}


% The telephone icon
\newcommand*{\phone}[2]{
 \begin{tikzpicture}
  \filldraw [fill=mymagenta, draw=black] (-3, 0) rectangle (3,2);
  \node at (0,1.4){\color{red!20!black}\huge #1};% The name of the router
  \node at (0,0.4){\color{red!20!black}\huge #2};% The IP address of the router
 \end{tikzpicture}
}

\newcommand*{\phonehidden}[2]{
 \begin{tikzpicture}
  \filldraw [fill=lightgray, draw=gray] (-3, 0) rectangle (3,2);
  \node at (0,1.4){\color{red!10!gray}\huge #1};% The name of the router
  \node at (0,0.4){\color{red!10!gray}\huge #2};% The IP address of the router
 \end{tikzpicture}
}
